% !TEX encoding = UTF-8
% !TEX TS-program = pdflatex
% !TEX root = ../Thesis.tex
% !TEX spellcheck = en-EN

%*******************************************************
% Abstract
%*******************************************************

% mi sono mantenuto generale senza parlare di assicurazioni, stima danno ed automobili

\chapter*{Abstract}
Quantificare il danno da rimborsare al cliente in caso di sinistro stradale è noto essere una pratica temporalmente ed economicamente dispendiosa per le compagnie assicurative.

L'obiettivo di questa tesi è quello di automatizzare il processo di stima economica del danno, attraverso l'elaborazione di immagini di sinistri stradali utilizzando tecniche di computer vision e deep learning. 

In particolare, sono state addestrate architetture di reti neurali per individuare la sezione dell'immagine contenente l'autovettura, l'identificazione delle componenti visibili, e la gravità di eventuali danni in modo da quantificare l'importo economico delle componenti danneggiate.

I risultati ottenuti sui dati storici hanno permesso la messa in produzione del processo automatico e l'avvio di una fase di validazione a supporto della procedura ordinaria.
