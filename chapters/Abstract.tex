% !TEX encoding = UTF-8
% !TEX TS-program = pdflatex
% !TEX root = ../Thesis.tex
% !TEX spellcheck = en-EN

%*******************************************************
% Abstract
%*******************************************************

% mi sono mantenuto generale senza parlare di assicurazioni, stima danno ed automobili

\chapter*{Abstract}
Quantificare il danno da rimborsare il cliente in caso di sinistro stradale, è noto essere una pratica temporalmente ed economicamente dispendiosa per le compagnie assicurative.

L'obiettivo di questa tesi è quello di automatizzare il processo di stima economica del danno, attraverso l'elaborazione di immagini di sinistri stradali utilizzando tecniche di computer vision e deep learning. 

In particolare sono state addestrate architetture di reti neurali per individuare la sezione dell'immagine contenente l'autovettura, l'identificazione delle componenti visibili, e la gravità di eventuali danni in modo da quantificare l'importo economico delle componenti daneggiate.

I risultati ottenuti sui dati storici hanno permesso la messa in produzione del processo automatico e l'avvio di una fase di validazione a supporto della procedura ordinaria.





For this master's thesis, a pipeline of Machine Learning models has been implemented in order to detect and classify objects in a set of images and use those as (for multi-instance learning?) different point of view to make a more accurate classification. An high-level view of the pipeline consists in filtering irrelevant incoming images, detecting and cropping relevant region of interest (car parts in this case) and classifying them in predetermined classes. Deep learning techniques have emerged in recent years as powerful methods for learning feature representations directly from data, and have led to remarkable breakthroughs in the field of Computer vision. In particular generic object detection, which aims at locating multiple objects instances from a large number of predefined categories, it is emerging as one of the most fundamental and challenging problems in this field. 

In section one and two an overview of the basics of Neural Networks and Deep Learning for Computer vision is given together with a review of the strengths and weaknesses of state of the art detectors and approaches for image classification and object detection. A more in-depth explanation of the archiecture is discussed in section three and finally an evualuation of the approach is presented in the last section.




input set immagini e output stima monetaria, motivazione, utilizzando
architetture di dep leraning in tre step ojdet, segm, classification

il processo di stima è stato un collo di bottiglia (parti dal problema e poi dici come lo risolvi)






% Given this time of rapid evolution, 