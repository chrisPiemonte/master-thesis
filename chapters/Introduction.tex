% !TEX encoding = UTF-8
% !TEX TS-program = pdflatex
% !TEX root = ../Tesi.tex
% !TEX spellcheck = it-IT

%*******************************************************
% Introduzione
%*******************************************************

\chapter*{Introduzione}

Il Deep Learning è un campo del Machine Learning nato nella metà del secolo scorso, ma che solo negli ultimi anni è diventato tra i più famosi ambiti di ricerca, portando ad una rivoluzione in gran parte dell'Intelligenza Artificiale, raggiungendo risultati paragonabili a quelli umani in alcuni casi.

I risultati notevoli degli ultimi anni hanno alzato di molto le aspettative su ciò che questo settore sarà in grado di raggiungere nel prossimo decennio. Nonostante applicazioni rivoluzionarie, come ad esempio le auto con guida autonoma, siano già a portata di mano, molte altre rimarranno probabilmente irraggiungibili per molto tempo, come ad esempio sistemi di dialogo credibili, traduzione automatica a livello umano attraverso due qualsiasi lingue e comprensione della lingua naturale (Natural Language Understanding) a livello umano. Il rischio con grandi aspettative a breve termine è che, con il fallimento della tecnologia, gli investimenti nella ricerca si prosciugheranno, rallentando i progressi a lungo, come già capitato in passato.

Tuttavia, i risultati ottenuti sono innegabili. Nel campo della Computer Vision, ad esempio, le Convolutional Neural Networks hanno completamente rimpiazzato i metodi classici. Introdotte inizialmente da Yann LeCun nel '98 \cite{lecun1998gradient}, e poi portate alla ribalta da Alex Krizhevsky e Geoffrey Hinton con la rete "AlexNet" \cite{krizhevsky2012imagenet} con cui vinsero la ImageNet competition portando la top-five accuracy degli anni precedenti da 74.3\% ad 83.6\%. Da allora, la competizione è stata dominata dalle Convolutional Neural Network. Nel 2015, il vincitore ha raggiunto un'accuratezza del 96,4\% e il compito di classificazione su ImageNet è considerato un problema risolto \cite{francois2017deep}.

La crescita del Deep Learning è stata dovuta alla loro abilità di automatizzare quello che era un passaggio cruciale: il feature engineering. Le precedenti tecniche di Machine Learning gestivano i dati input solitamente tramite proiezioni non lineari o altre trasformazioni semplici. Ma la rappresentazione di problemi complessi può non essere raggiunta da tali tecniche. Pertanto, era necessario rendere i dati di input più rappresentativi attraverso regole manuali. Il Deep Learning, d'altra parte, automatizza completamente questo passaggio, apprendendo le combinazioni dei dati in input più rappresentative congiuntamente all'apprendimento del task senza doverle progettare personalmente. Ciò ha notevolmente semplificato i flussi di lavoro di Machine Learning, sostituendo sofisticate pipeline con modelli di Deep Learning, addestrabili e end-to-end.

Dal 2012, le Convolutional Neural Networks (convnets) sono diventate lo standard per tutte le attività di Computer Vision. Allo stesso tempo, il Deep Learning ha trovato applicazioni in molti altri tipi di problemi, come nel Natural Language processing, sostituendo completamente Support Vector Machines (SVM) e Decision Trees in una vasta gamma di applicazioni.

